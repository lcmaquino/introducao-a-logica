\documentclass[12pt,a4paper]{article}
\usepackage[utf8]{inputenc}
\usepackage[brazil]{babel}
\usepackage{graphicx}
\usepackage{amssymb, amsfonts, amsmath}
\usepackage{float}
\usepackage{enumerate}
\usepackage[top=2.5cm, bottom=2.5cm, left=1.25cm, right=1.25cm]{geometry}

\begin{document}
\pagestyle{empty}

\begin{center}
  \begin{tabular}{ccc}
    \begin{tabular}{c}
      \includegraphics[scale=0.25]{../../biblioteca/imagem/brasao-de-armas-brasil} \\
    \end{tabular} & 
    \begin{tabular}{c}
      Ministério da Educação \\
      Universidade Federal dos Vales do Jequitinhonha e Mucuri \\
      Faculdade de Ciências Sociais, Aplicadas e Exatas - FACSAE \\
      Departamento de Ciências Exatas - DCEX \\
      Disciplina: Introdução à Lógica \quad Semestre: 2021/2\\
      Prof. Me. Luiz C. M. de Aquino\\
    \end{tabular} &
    \begin{tabular}{c}
      \includegraphics[scale=0.25]{../../biblioteca/imagem/logo-ufvjm} \\
    \end{tabular}
  \end{tabular}
\end{center}

\begin{center}
  \textbf{Lista IV}
\end{center}

\begin{enumerate}
  \item Monte a tabela verdade para verificar se cada argumento abaixo é válido ou inválido.
  \begin{enumerate}
    \item $p\to q,\, r\to \lnot q \vdash r \to \lnot p$
    \item $p \wedge \lnot q,\, \lnot r \to q \vdash p\wedge r$
    \item $p\to \lnot q,\, p,\, q\to r \vdash \lnot r$
    \item $p\vee \lnot q,\, \lnot p,\, \lnot (p\wedge r)\to q \vdash r$
  \end{enumerate}
  
  \item Passe cada argumento abaixo para a forma simbólica e verifique a sua validade.

    \begin{tabular}{ll}
      (a) & 
      \begin{tabular}{l}
        Se eu trabalho, então eu recebo remuneração.\\
        Eu recebo remuneração e eu como feijão.\\
        Eu não trabalho.\\ \hline
        Logo, eu não como feijão.
      \end{tabular}
    \end{tabular}

    \begin{tabular}{ll}
      (b) & 
      \begin{tabular}{l}
        Se eu estudo e eu trabalho, então eu otimizo o meu tempo.\\
        Se eu tenho planos para o futuro, então eu estudo e eu trabalho.\\
        Eu tenho planos para o futuro.\\ \hline
        Logo, eu trabalho ou eu otimizo o meu tempo.
      \end{tabular}
    \end{tabular}
 

    \begin{tabular}{ll}
      (c) & 
      \begin{tabular}{l}
        Se eu sou uma ave, então eu bebo água.\\
        Se eu sou bípede, então eu tenho um bico.\\
        Eu sou uma ave ou eu tenho um bico.\\ \hline
        Logo, eu bebo água ou eu sou bípede.
      \end{tabular}
    \end{tabular}

    \begin{tabular}{ll}
      (d) & 
      \begin{tabular}{l}
        Seu eu jogo videogame ou eu pratico esporte, então eu não tomo refrigerante.\\
        Eu jogo videogame.\\
        Se eu tomo suco, então eu tomo refrigerante.\\ \hline
        Logo, eu não tomo suco.
      \end{tabular}
    \end{tabular}

  \item Dê exemplo de um argumento válido que tenha 4 premissas e monte a sua tabela verdade.
  
  \item Dê exemplo de um argumento inválido que tenha 4 premissas e monte a sua tabela verdade.
 
\end{enumerate}

\end{document}