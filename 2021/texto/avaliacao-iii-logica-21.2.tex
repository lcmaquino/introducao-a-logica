\documentclass[12pt,a4paper]{article}
\usepackage[utf8]{inputenc}
\usepackage[brazil]{babel}
\usepackage{graphicx}
\usepackage{amssymb, amsfonts, amsmath}
\usepackage{float}
\usepackage{enumerate}
\usepackage[top=2.5cm, bottom=2.5cm, left=1.25cm, right=1.25cm]{geometry}

\begin{document}
\pagestyle{empty}

\begin{center}
  \begin{tabular}{ccc}
    \begin{tabular}{c}
      \includegraphics[scale=0.25]{../../biblioteca/imagem/brasao-de-armas-brasil} \\
    \end{tabular} & 
    \begin{tabular}{c}
      Ministério da Educação \\
      Universidade Federal dos Vales do Jequitinhonha e Mucuri \\
      Faculdade de Ciências Sociais, Aplicadas e Exatas - FACSAE \\
      Departamento de Ciências Exatas - DCEX \\
      Disciplina: Introdução à Lógica \quad Semestre: 2021/2\\
      Prof. Me. Luiz C. M. de Aquino\\
    \end{tabular} &
    \begin{tabular}{c}
      \includegraphics[scale=0.25]{../../biblioteca/imagem/logo-ufvjm} \\
    \end{tabular}
  \end{tabular}
\end{center}

\begin{center}
 \textbf{Avaliação III}
\end{center}

\textbf{Instruções}
\begin{itemize}
 \item Todas as justificativas necessárias na solução de cada questão devem estar presentes nesta avaliação;
 \item As respostas finais de cada questão devem estar escritas de caneta;
 \item Esta avaliação tem um total de 35,0 pontos.
\end{itemize}

\begin{enumerate}
  \item \textbf{[8,0 pontos]} Monte a tabela verdade para verificar se cada argumento abaixo é válido ou inválido.
  \begin{enumerate}
    \item $p\to q,\, r\to \lnot q \vdash r \to \lnot p$
    \item $p\to \lnot q,\, p,\, q\to r \vdash \lnot r$
  \end{enumerate}
  
  \item \textbf{[9,0 pontos]} Passe cada argumento abaixo para a forma simbólica e verifique a sua validade.

    \begin{tabular}{ll}
      (a) & 
      \begin{tabular}{l}
        Se eu estudo e eu trabalho, então eu otimizo o meu tempo.\\
        Se eu tenho planos para o futuro, então eu estudo e eu trabalho.\\
        Eu tenho planos para o futuro.\\ \hline
        Logo, eu trabalho ou eu otimizo o meu tempo.
      \end{tabular}
    \end{tabular}
 
    \begin{tabular}{ll}
      (b) & 
      \begin{tabular}{l}
        Seu eu jogo videogame ou eu pratico esporte, então eu não tomo refrigerante.\\
        Eu jogo videogame.\\
        Se eu tomo suco, então eu tomo refrigerante.\\ \hline
        Logo, eu não tomo suco.
      \end{tabular}
    \end{tabular}

  \item \textbf{[9,0 pontos]}  Dê exemplo de um argumento válido que tenha 4 premissas 
  e monte a sua tabela verdade.

  \item \textbf{[9,0 pontos]} Dê exemplo de um argumento inválido que tenha 4 premissas 
  e monte a sua tabela verdade.

\end{enumerate}

\end{document}