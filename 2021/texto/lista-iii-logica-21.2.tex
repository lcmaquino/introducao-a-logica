\documentclass[12pt,a4paper]{article}
\usepackage[utf8]{inputenc}
\usepackage[brazil]{babel}
\usepackage{graphicx}
\usepackage{amssymb, amsfonts, amsmath}
\usepackage{float}
\usepackage{enumerate}
\usepackage[top=2.5cm, bottom=2.5cm, left=1.25cm, right=1.25cm]{geometry}

\begin{document}
\pagestyle{empty}

\begin{center}
  \begin{tabular}{ccc}
    \begin{tabular}{c}
      \includegraphics[scale=0.25]{../../biblioteca/imagem/brasao-de-armas-brasil} \\
    \end{tabular} & 
    \begin{tabular}{c}
      Ministério da Educação \\
      Universidade Federal dos Vales do Jequitinhonha e Mucuri \\
      Faculdade de Ciências Sociais, Aplicadas e Exatas - FACSAE \\
      Departamento de Ciências Exatas - DCEX \\
      Disciplina: Introdução à Lógica \quad Semestre: 2021/2\\
      Prof. Me. Luiz C. M. de Aquino\\
    \end{tabular} &
    \begin{tabular}{c}
      \includegraphics[scale=0.25]{../../biblioteca/imagem/logo-ufvjm} \\
    \end{tabular}
  \end{tabular}
\end{center}

\begin{center}
  \textbf{Lista III}
\end{center}

\begin{enumerate}
  \item Escreva um contraexemplo para as proposições abaixo.
  \begin{enumerate}
    \item A subtração entre números naturais é um número natural.
    \item A soma entre números irracionais é um número irracional.
    \item Se $x\in\mathbb{R}$ e $x^2$ é par, então $x$ é par.
    \item Se $\dfrac{a}{b} = \dfrac{2}{5}$, então $a = 2$ e $b = 5$.
  \end{enumerate}
  
  \item Prove de forma direta as proposições abaixo.
  \begin{enumerate}
    \item A soma entre números racionais é um número racional.
    \item A subtração entre números racionais é um número racional.
    \item A multiplicação entre números racionais é um número racional.
    \item A divisão entre números racionais é um número racional.
  \end{enumerate}
  
  \item Prove por absurdo as proposições abaixo.
  \begin{enumerate}
    \item A soma entre um número racional e um número irracional será um número irracional.
    \item A subtração entre um número racional e um número irracional será um número irracional.
    \item Se $x,\,y\in\mathbb{R}$ e $xy = 0$, então $x = 0$ ou $y = 0$.
    \item $\sqrt{3}$ é irracional.
  \end{enumerate}

  \item Prove usando o Princípio de Indução Finita (PIF) as proposições abaixo.
  \begin{enumerate}
    \item $5^1 + 5^2 + 5^3 + \ldots + 5^n = \dfrac{5}{4}\left(5^{n} - 1\right)$.
    \item $7^n - 1$ é divisível por 2.
    \item $2^n > n$.
    \item $\dfrac{a^n - 1}{a - 1} = a^{n - 1} + a^{n - 2} + a^{n - 3} + \ldots + a + 1$.
  \end{enumerate}
  
\end{enumerate}

\end{document}