\documentclass[12pt,a4paper]{article}
\usepackage[utf8]{inputenc}
\usepackage[brazil]{babel}
\usepackage{graphicx}
\usepackage{amssymb, amsfonts, amsmath}
\usepackage{float}
\usepackage{enumerate}
\usepackage[top=2.5cm, bottom=2.5cm, left=1.25cm, right=1.25cm]{geometry}

\begin{document}
\pagestyle{empty}

\begin{center}
  \begin{tabular}{ccc}
    \begin{tabular}{c}
      \includegraphics[scale=0.25]{../../biblioteca/imagem/brasao-de-armas-brasil} \\
    \end{tabular} & 
    \begin{tabular}{c}
      Ministério da Educação \\
      Universidade Federal dos Vales do Jequitinhonha e Mucuri \\
      Faculdade de Ciências Sociais, Aplicadas e Exatas - FACSAE \\
      Departamento de Ciências Exatas - DCEX \\
      Disciplina: Introdução à Lógica \quad Semestre: 2021/2\\
      Prof. Me. Luiz C. M. de Aquino\\
    \end{tabular} &
    \begin{tabular}{c}
      \includegraphics[scale=0.25]{../../biblioteca/imagem/logo-ufvjm} \\
    \end{tabular}
  \end{tabular}
\end{center}

\begin{center}
  \textbf{Lista II}
\end{center}

\begin{enumerate}
  \item Construa a tabela verdade das proposições abaixo.
 
    \begin{enumerate}
      \item $(p \leftrightarrow (\lnot q))\rightarrow((\lnot p) \vee q)$
      \item $(r \rightarrow (p \wedge (\lnot q)))\vee ((\lnot r) \vee (p \leftrightarrow (\lnot p)))$
    \end{enumerate}

  \item Considerando que o valor lógico de $p$ e $q$ é verdadeiro e de $r$ é falso, determine o 
    valor lógico das proposições abaixo.

    \begin{enumerate}
      \item $(p \wedge (\lnot q))\rightarrow ((\lnot r) \leftrightarrow p)$
      \item $((\lnot p) \vee (\lnot q)) \,\underline{\vee}\, ((\lnot r)\rightarrow p)$
      \item $(p \,\underline{\vee}\, (\lnot p))\wedge ((r \leftrightarrow (\lnot q))\wedge ((\lnot p)\rightarrow q)$
      \item $((\lnot r)\rightarrow q) \wedge (p \vee (r \rightarrow (\lnot p)))$
    \end{enumerate}

  \item Considerando a proposição condicional ``se eu estudo matemática, então eu sou maluco'', escreva 
  as suas respectivas proposições:

    \begin{enumerate}
      \item recíproca;
      \item contrária;
      \item contrapositiva;
      \item negação.
    \end{enumerate}

  \item Renato, Luiz e Marcelo são casados com Ana, Christiane e Francislene, não necessariamente nesta ordem. 
    Um dos maridos é professor, outro é engenheiro e outro, empresário. Com base nas dicas abaixo, determine 
    a profissão de cada um e o nome de sua respectiva esposa.
    
  \begin{itemize}
    \item O empresário é casado com Ana.
    \item Luiz é professor.
    \item Christiane não é casada com Luiz.
    \item Renato não é empresário.
  \end{itemize}

  \item (Vunesp/TJ-SP - Adaptada) Em um edifício com apartamentos somente nos andares de 1º ao 4º, moram 4 meninas, 
    em andares distintos: Juliana, Carla, Alice e Maria, não necessariamente nessa ordem. 
    Cada uma delas tem um animal de estimação diferente: gato, cachorro, passarinho e tartaruga, 
    não necessariamente nessa ordem. Maria vive reclamando do barulho feito pelo cachorro, 
    no andar imediatamente acima do seu. Juliana, que não mora no 4º andar, mora um andar acima do de Alice, 
    que tem o passarinho e não mora no 2º andar. Quem mora no 3º andar tem uma tartaruga. 
    Determine em qual andar mora cada menina e qual seu respectivo animal de estimação. 

\end{enumerate}

\newpage

\begin{scriptsize}
\begin{center}
  \textbf{Gabarito}
\end{center}

[1]
 
(a)

\begin{tabular}{|c|c|c|c|c|c|c|}
\hline 
 $p$ & $q$ & $\lnot p$ & $\lnot q$ & $p \leftrightarrow (\lnot q)$ & $(\lnot p) \vee q$ & $(p \leftrightarrow (\lnot q))\rightarrow((\lnot p) \vee q)$ \\ \hline
V & V & F & F & F & V & V\\ \hline
V & F & F & V & V & F & F\\ \hline
F & V & V & F & V & V & V\\ \hline
F & F & V & V & F & V & V\\ \hline
\end{tabular}

\vspace{10pt}

(b)

  \begin{tabular}{|c|c|c|c|c|c|c|c|c|c|c|}
    \hline 
      $p$ & $q$ & $r$ & $\lnot p$ & $\lnot q$ & $\lnot r$ & $p \wedge (\lnot q)$ & $r\rightarrow (p \wedge (\lnot q))$ & $p \leftrightarrow (\lnot p)$ & $(\lnot r)\vee (p \leftrightarrow (\lnot p))$ & $(r \rightarrow (p \wedge (\lnot q)))\vee ((\lnot r) \vee (p \leftrightarrow (\lnot p)))$\\ \hline
      V & V & V & F & F & F & F & F & F & F & F\\ \hline
      V & V & F & F & F & V & F & V & F & V & V\\ \hline
      V & F & V & F & V & F & V & V & F & F & V\\ \hline
      V & F & F & F & V & V & V & V & F & V & V\\ \hline
      F & V & V & V & F & F & F & F & F & F & F\\ \hline
      F & V & F & V & F & V & F & V & F & V & V\\ \hline
      F & F & V & V & V & F & F & F & F & F & F\\ \hline
      F & F & F & V & V & V & F & V & F & V & V\\ \hline
  \end{tabular}

\vspace{10pt}

[2] 

(a) F. 

(b) V. 

(c) V. 

(d) V.

\vspace{10pt}

[3] 

(a) ``Se eu sou maluco, então eu estudo matemática''.

(b) ``Se eu não estudo matemática, então eu não sou maluco''.

(c) ``Se eu não sou maluco, então eu não estudo matemática''.

(d) ``Eu estudo matemática e eu não sou maluco''.

\vspace{10pt}

[4] Luiz é professor e casado com Francislene. 
Renato é engenheiro e casado com Christiane. 
Marcelo é empresário e casado com Ana.

\vspace{10pt}

[5] 
Alice mora no 1º andar e tem um passarinho. 
Juliana mora no 2º andar e tem um gato. 
Maria mora no 3º andar e tem um tartaruga.  
Carla mora no 4º andar e tem um cachorro. 

\end{scriptsize}

\end{document}