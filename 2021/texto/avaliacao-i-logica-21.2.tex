\documentclass[12pt,a4paper]{article}
\usepackage[utf8]{inputenc}
\usepackage[brazil]{babel}
\usepackage{graphicx}
\usepackage{amssymb, amsfonts, amsmath}
\usepackage{float}
\usepackage{enumerate}
\usepackage[top=2.5cm, bottom=2.5cm, left=1.25cm, right=1.25cm]{geometry}

\begin{document}
\pagestyle{empty}

\begin{center}
  \begin{tabular}{ccc}
    \begin{tabular}{c}
      \includegraphics[scale=0.25]{../../biblioteca/imagem/brasao-de-armas-brasil} \\
    \end{tabular} & 
    \begin{tabular}{c}
      Ministério da Educação \\
      Universidade Federal dos Vales do Jequitinhonha e Mucuri \\
      Faculdade de Ciências Sociais, Aplicadas e Exatas - FACSAE \\
      Departamento de Ciências Exatas - DCEX \\
      Disciplina: Introdução à Lógica \quad Semestre: 2021/2\\
      Prof. Me. Luiz C. M. de Aquino\\
    \end{tabular} &
    \begin{tabular}{c}
      \includegraphics[scale=0.25]{../../biblioteca/imagem/logo-ufvjm} \\
    \end{tabular}
  \end{tabular}
\end{center}

\begin{center}
 \textbf{Avaliação I}
\end{center}

\textbf{Instruções}
\begin{itemize}
 \item Todas as justificativas necessárias na solução de cada questão devem estar presentes nesta avaliação;
 \item As respostas finais de cada questão devem estar escritas de caneta;
 \item Esta avaliação tem um total de 30,0 pontos.
\end{itemize}

\begin{enumerate}
  \item \textbf{[6,0 pontos]} Complete a tabela verdade abaixo.
  
  \begin{center}
    \begin{tabular}{|c|c|c|c|c|c|c|c|c|c|}
      \hline
      $p$ & $q$ & $r$ & $\lnot p$ & $\lnot q$ & $\lnot r$ &  $(\lnot r) \leftrightarrow (\lnot q)$ & $(\lnot p)\rightarrow q$ & $((\lnot r) \leftrightarrow (\lnot q))\wedge ((\lnot p)\rightarrow q)$\\ \hline
       & & & & & & & & \\ \hline
       & & & & & & & &  \\ \hline
       & & & & & & & &  \\ \hline
       & & & & & & & &  \\ \hline
       & & & & & & & &  \\ \hline
       & & & & & & & &  \\ \hline
       & & & & & & & &  \\ \hline
       & & & & & & & &  \\ \hline
    \end{tabular}
  \end{center}

  \item \textbf{[6,0 pontos]} Determine o valor lógico das proposições abaixo.

    \begin{enumerate}
      \item $(2 < 6) \wedge (0 \geq 1)$
      \item $\left(\pi \leq \dfrac{1}{3}\right)\to (1 + 1 \neq 2)$
      \item $\left(0 \not\in \mathbb{R}\right)\vee \left(0 = \dfrac{1}{1}\right)$
      \item $\left(12 = 3\cdot 4\right) \leftrightarrow \left(12 = 10 + 2\right)$
    \end{enumerate}
      
  \item \textbf{[6,0 pontos]} Considerando que o valor lógico de $p$ e $q$ é verdadeiro e de $r$ é falso, determine o 
    valor lógico das proposições abaixo.

    \begin{enumerate}
      \item $(p \,\underline{\vee}\, (\lnot p))\wedge ((r \leftrightarrow (\lnot q))\wedge ((\lnot p)\rightarrow q)$
      \item $((\lnot r)\rightarrow q) \wedge (p \vee (r \rightarrow (\lnot p)))$
    \end{enumerate}

  \item \textbf{[6,0 pontos]} Considerando a proposição condicional ``se eu gosto de banana, então eu não gosto de maça'', escreva 
  as suas respectivas proposições:

    \begin{enumerate}
      \item recíproca;
      \item contrária;
      \item contrapositiva;
      \item negação.
    \end{enumerate}

  \item \textbf{[6,0 pontos]} Renato, Luiz e Marcelo são casados com Ana, Christiane e Francislene, não necessariamente nesta ordem. 
    Um dos maridos é professor, outro é engenheiro e outro, empresário. Com base nas dicas abaixo, determine 
    a profissão de cada um e o nome de sua respectiva esposa.
    
  \begin{itemize}
    \item O empresário é casado com Ana.
    \item Luiz é professor.
    \item Christiane não é casada com Luiz.
    \item Renato não é empresário.
  \end{itemize}

\end{enumerate}

\end{document}