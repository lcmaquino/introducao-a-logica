\documentclass[12pt,a4paper]{article}
\usepackage[utf8]{inputenc}
\usepackage[brazil]{babel}
\usepackage{graphicx}
\usepackage{amssymb, amsfonts, amsmath}
\usepackage{float}
\usepackage{enumerate}
\usepackage[top=2.5cm, bottom=2.5cm, left=1.25cm, right=1.25cm]{geometry}

\begin{document}
\pagestyle{empty}

\begin{center}
  \begin{tabular}{ccc}
    \begin{tabular}{c}
      \includegraphics[scale=0.25]{../../biblioteca/imagem/brasao-de-armas-brasil} \\
    \end{tabular} & 
    \begin{tabular}{c}
      Ministério da Educação \\
      Universidade Federal dos Vales do Jequitinhonha e Mucuri \\
      Faculdade de Ciências Sociais, Aplicadas e Exatas - FACSAE \\
      Departamento de Ciências Exatas - DCEX \\
      Disciplina: Introdução à Lógica \quad Semestre: 2021/2\\
      Prof. Me. Luiz C. M. de Aquino\\
    \end{tabular} &
    \begin{tabular}{c}
      \includegraphics[scale=0.25]{../../biblioteca/imagem/logo-ufvjm} \\
    \end{tabular}
  \end{tabular}
\end{center}

\begin{center}
  \textbf{Lista I}
\end{center}

\begin{enumerate}
  \item Quatro suspeitos de praticar um crime fazem as seguintes declarações:
  
    \begin{itemize}
      \item João: Carlos é o criminoso;
      \item Pedro: eu não sou criminoso;
      \item Carlos: Paulo é o criminoso;
      \item Paulo: Carlos está mentindo;
    \end{itemize}

  Sabendo que APENAS UM dos suspeitos mente, determine quem é o criminoso.

    \begin{enumerate}
      \item João.
      \item Pedro.
      \item Carlos.
      \item Paulo.
    \end{enumerate}

  \item (FGV/TCE-SE -- Adaptada) Considere a afirmação: ``Se hoje é sábado, então amanhã não trabalharei''. 
  A negação dessa afirmação é:

    \begin{enumerate}
      \item Hoje é sábado e amanhã trabalharei.
      \item Hoje não é sábado e amanhã trabalharei.
      \item Hoje não é sábado ou amanhã trabalharei.
      \item Se hoje não é sábado, então amanhã trabalharei.
      \item Se hoje não é sábado, então amanhã não trabalharei.
    \end{enumerate}

  \item (FGV – TJSC – 2015) Considere a sentença: ``Se cometi um crime, então
    serei condenado''. Uma sentença logicamente equivalente à sentença dada é:

    \begin{enumerate}
      \item  Não cometi um crime ou serei condenado.
      \item  Se não cometi um crime, então não serei condenado.
      \item  Se eu for condenado, então cometi um crime.
      \item  Cometi um crime e serei condenado.
      \item  Não cometi um crime e não serei condenado.
    \end{enumerate}

  \item (CESPE – TRE/MT – 2015) A negação da proposição: ``Se o número
    inteiro $m > 2$ é primo, então o número $m$ é ímpar'' pode ser expressa
    corretamente por:

    \begin{enumerate}
      \item ``Se o número $m$ não é ímpar, então o número inteiro $m > 2$ não é primo''.
      \item ``Se o número inteiro $m > 2$ não é primo, então o número $m$ é ímpar''.
      \item ``O número inteiro $m > 2$ é primo e o número $m$ não é ímpar''.
      \item ``O número inteiro $m > 2$ é não primo e o número $m$ é ímpar''.
      \item ``Se o número inteiro $m > 2$ não é primo, então o número $m$ não é ímpar''
    \end{enumerate}
    
  \item (FCC – SABESP – 2014) Alan, Beto, Caio e Décio são irmãos e foram
    interrogados pela própria mãe para saber quem comeu, sem autorização,
    o chocolate que estava no armário. Sabe-se que apenas um dos quatro
    comeu o chocolate, e que os quatro irmãos sabem quem foi. A mãe
    perguntou para cada um quem cometeu o ato, ao que recebeu as seguintes
    respostas:

    \begin{itemize}   
      \item Alan diz que foi Beto;
      \item Beto diz que foi Caio;
      \item Caio diz que Beto mente;
      \item Décio diz que não foi ele.
    \end{itemize}

    O irmão que fala a verdade e o irmão que comeu o chocolate são,
    respectivamente,

    \begin{enumerate}
      \item Beto e Décio.
      \item Alan e Beto.
      \item Beto e Caio.
      \item Alan e Caio.
      \item Caio e Décio
    \end{enumerate}
        
        
    \item Alice, Ana Lúcia e Rafael são netos(as) da vovó Maria. Apenas 
    um(a) deles(as) quebrou o vaso de vidro da vovó Maria. E apenas um(a) 
    deles(as) está falando a verdade abaixo.

    \begin{itemize}   
      \item Alice: O Rafael está falando a verdade.
      \item Ana Lúcia: O Rafael e a Alice estão mentindo.
      \item Rafael: Eu não quebrei o vaso de vidro da vovó Maria.
    \end{itemize}
    
    Quem quebrou o vaso de vidro da vovó Maria?

\end{enumerate}

\begin{center}
  \textbf{Gabarito}
\end{center}

[1] (c). 
[2] (a). 
[3] (a).
[4] (c). 
[5] (e). 
[6] Rafael.

\end{document}