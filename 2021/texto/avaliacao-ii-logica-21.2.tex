\documentclass[12pt,a4paper]{article}
\usepackage[utf8]{inputenc}
\usepackage[brazil]{babel}
\usepackage{graphicx}
\usepackage{amssymb, amsfonts, amsmath}
\usepackage{float}
\usepackage{enumerate}
\usepackage[top=2.5cm, bottom=2.5cm, left=1.25cm, right=1.25cm]{geometry}

\begin{document}
\pagestyle{empty}

\begin{center}
  \begin{tabular}{ccc}
    \begin{tabular}{c}
      \includegraphics[scale=0.25]{../../biblioteca/imagem/brasao-de-armas-brasil} \\
    \end{tabular} & 
    \begin{tabular}{c}
      Ministério da Educação \\
      Universidade Federal dos Vales do Jequitinhonha e Mucuri \\
      Faculdade de Ciências Sociais, Aplicadas e Exatas - FACSAE \\
      Departamento de Ciências Exatas - DCEX \\
      Disciplina: Introdução à Lógica \quad Semestre: 2021/2\\
      Prof. Me. Luiz C. M. de Aquino\\
    \end{tabular} &
    \begin{tabular}{c}
      \includegraphics[scale=0.25]{../../biblioteca/imagem/logo-ufvjm} \\
    \end{tabular}
  \end{tabular}
\end{center}

\begin{center}
 \textbf{Avaliação II}
\end{center}

\textbf{Instruções}
\begin{itemize}
 \item Todas as justificativas necessárias na solução de cada questão devem estar presentes nesta avaliação;
 \item As respostas finais de cada questão devem estar escritas de caneta;
 \item Esta avaliação tem um total de 35,0 pontos.
\end{itemize}

\begin{enumerate}
  \item \textbf{[5,0 pontos]} Escreva um contraexemplo para as proposições abaixo.
  \begin{enumerate}
    \item A subtração entre números naturais é um número natural.
    \item A soma entre números irracionais é um número irracional.
    \item Se $x\in\mathbb{R}$ e $x^2$ é par, então $x$ é par.
    \item Se $\dfrac{a}{b} = \dfrac{2}{5}$, então $a = 2$ e $b = 5$.
  \end{enumerate}
  
  \item \textbf{[8,0 pontos]} Prove de forma direta as proposições abaixo.
  \begin{enumerate}
    \item A soma entre números racionais é um número racional.
    \item A multiplicação entre números racionais é um número racional.
  \end{enumerate}
  
  \item \textbf{[8,0 pontos]} Prove por absurdo as proposições abaixo.
  \begin{enumerate}
    \item Se $x,\,y\in\mathbb{R}$ e $xy = 0$, então $x = 0$ ou $y = 0$.
    \item $\sqrt{3}$ é irracional.
  \end{enumerate}

  \item \textbf{[8,0 pontos]} Prove usando o Princípio de Indução Finita (PIF) as proposições abaixo.
  \begin{enumerate}
    \item $7^n - 1$ é divisível por 2.
    \item $\dfrac{a^n - 1}{a - 1} = a^{n - 1} + a^{n - 2} + a^{n - 3} + \ldots + a + 1$.
  \end{enumerate}

  \item \textbf{[6,0 pontos]} Prove que se $a$ e $a + b$ são divisíveis por $c$, então $b$ é divisível 
  por $c$.

\end{enumerate}

\end{document}