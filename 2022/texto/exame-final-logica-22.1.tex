\documentclass[12pt,a4paper]{article}
\usepackage[utf8]{inputenc}
\usepackage[brazil]{babel}
\usepackage{graphicx}
\usepackage{amssymb, amsfonts, amsmath}
\usepackage{float}
\usepackage{enumerate}
\usepackage[top=2.5cm, bottom=2.5cm, left=1.25cm, right=1.25cm]{geometry}

\begin{document}
\pagestyle{empty}

\begin{center}
  \begin{tabular}{ccc}
    \begin{tabular}{c}
      \includegraphics[scale=0.25]{../../biblioteca/imagem/brasao-de-armas-brasil} \\
    \end{tabular} & 
    \begin{tabular}{c}
      Ministério da Educação \\
      Universidade Federal dos Vales do Jequitinhonha e Mucuri \\
      Faculdade de Ciências Sociais, Aplicadas e Exatas - FACSAE \\
      Departamento de Ciências Exatas - DCEX \\
      Disciplina: Introdução à Lógica \quad Semestre: 2022/1\\
      Prof. Dr. Luiz C. M. de Aquino\\
      Aluno(a):\rule{6cm}{0.1mm} \quad Data: \rule{0.5cm}{0.1mm}/\rule{0.5cm}{0.1mm}/\rule{1cm}{0.1mm}\\
    \end{tabular} &
    \begin{tabular}{c}
      \includegraphics[scale=0.25]{../../biblioteca/imagem/logo-ufvjm} \\
    \end{tabular}
  \end{tabular}
\end{center}

\begin{center}
 \textbf{Exame final}
\end{center}

\textbf{Instruções}
\begin{itemize}
 \item Todas as justificativas necessárias na solução de cada questão devem 
 estar presentes nesta avaliação;
 \item As respostas finais de cada questão devem estar escritas de caneta;
 \item Esta avaliação tem um total de 100,0 pontos.
\end{itemize}

\begin{enumerate}
  \item \textbf{[20,0 pontos]} Complete a tabela verdade abaixo.

  \begin{center}
    \begin{tabular}{|c|c|c|c|c|c|c|c|c|c|}
      \hline
      $p$ & $q$ & $r$ & $\lnot p$ & $\lnot q$ & 
      $\lnot q$ & $(\lnot r) \rightarrow (\lnot q)$ & 
      $(\lnot p)\leftrightarrow q$ & 
      $((\lnot r) \rightarrow (\lnot q))\vee ((\lnot p)\leftrightarrow q)$\\ \hline
       & & & & & & & & \\ \hline
       & & & & & & & &  \\ \hline
       & & & & & & & &  \\ \hline
       & & & & & & & &  \\ \hline
       & & & & & & & &  \\ \hline
       & & & & & & & &  \\ \hline
       & & & & & & & &  \\ \hline
       & & & & & & & &  \\ \hline
    \end{tabular}
  \end{center}

  \item \textbf{[20,0 pontos]} Escreva a negação de cada proposição composta abaixo:

    \begin{enumerate}
      \item Eu não gosto de sorvete de baunilha e eu gosto de doce de batata;

      \rule{\linewidth}{0.1mm}
      
      \rule{\linewidth}{0.1mm}
            
      \item Eu tenho 20 anos ou eu não estou solteiro;

      \rule{\linewidth}{0.1mm}
      
      \rule{\linewidth}{0.1mm}

      \item Eu faço pão se, e somente se, eu faço manteiga;

      \rule{\linewidth}{0.1mm}
      
      \rule{\linewidth}{0.1mm}
      
      \item Se eu não vou cozinhar feijão, então eu vou cozinhar beterraba.

      \rule{\linewidth}{0.1mm}
      
      \rule{\linewidth}{0.1mm}
    \end{enumerate}

  \item \textbf{[20,0 pontos]} Complete a tabela verdade para verificar se o argumento abaixo é válido ou inválido.
    \begin{enumerate}
      \item $p\to q,\, r\to \lnot q \vdash r \to \lnot p$

    \begin{center}
      \begin{tabular}{|c|c|c|c|c|c|c|c|c|c|}
        \hline
        $p$ & $q$ & $r$ & $\lnot p$ & $\lnot q$ & 
        $p \rightarrow q$ & $r \rightarrow (\lnot q)$ & 
        $r \rightarrow (\lnot p)$\\ \hline
         & & & & & & & \\ \hline
         & & & & & & & \\ \hline
         & & & & & & & \\ \hline
         & & & & & & & \\ \hline
         & & & & & & & \\ \hline
         & & & & & & & \\ \hline
         & & & & & & & \\ \hline
         & & & & & & & \\ \hline
      \end{tabular}
    \end{center}

  \end{enumerate}

  \item \textbf{[20,0 pontos]} Escreva um contraexemplo para as proposições abaixo.
    \begin{enumerate}
      \item A subtração entre números naturais é um número natural.
      \item A soma entre números irracionais é um número irracional.
      \item Se $x\in\mathbb{R}$ e $x^2$ é ímpar, então $x$ é ímpar.
      \item Se $\dfrac{a}{b} = \dfrac{7}{8}$, então $a = 7$ e $b = 8$.
    \end{enumerate}
 
  \item \textbf{[20,0 pontos]} Prove usando o Princípio de Indução Finita (PIF) a proposição abaixo.
    \begin{enumerate}
      \item $6^1 + 6^2 + 6^3 + \ldots + 6^n = \dfrac{6}{5}\left(6^{n} - 1\right)$.
    \end{enumerate}

\end{enumerate}

\end{document}